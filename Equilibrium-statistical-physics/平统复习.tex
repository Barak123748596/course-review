\documentclass[a4paper,12pt]{article}
\usepackage{CJKutf8}
\usepackage{indentfirst}
\usepackage{graphicx}
\usepackage{amsmath}
\usepackage{longtable}
\usepackage{fancyhdr}
\usepackage{multirow}
\usepackage{setspace}
\usepackage{textcomp}
\usepackage{mathrsfs}
\usepackage{float}
\usepackage{geometry}
\usepackage{braket}

\newcommand{\sj}{\quad\!\!\!\!\:}
\newcommand{\alg}[1]{\begin{align*}{#1}\end{align*}}
\newcommand{\myPartial}[3]{{\bigg(\frac{\partial {#1}}{\partial {#2}}\bigg)}_{#3}}


\pagestyle{fancy}
\setlength{\textwidth}{159.2mm}
\setlength{\oddsidemargin}{0pt}
\setlength{\voffset}{-10mm}
\setlength{\headwidth}{159.2mm}
\setlength{\textheight}{235mm}
\fancyhf{}
\lhead{\begin{CJK*}{UTF8}{gkai}平衡态统计物理\ 复习\end{CJK*}}
\rhead{\thepage}

\begin{document}
\begin{CJK*}{UTF8}{gbsn}

\title{\textbf{平衡态统计物理\\复习}}
\author{葛博文\;1500019707}
\date{\number\year年\number\month月\number\day日}
\maketitle

\begin{spacing}{1.3}
\setcounter{section}{0}
\section{热力学定律}
\begin{align}
  \alpha &= \frac{1}{V}\myPartial{V}{T}{p}\\
  \beta &= \frac{1}{p}\myPartial{p}{T}{V}\\
  \kappa_T &= -\frac{1}{V}\myPartial{V}{p}{T}\\
  S &= nC_p lnT - nRlnp +S_0 \Rightarrow \Delta S= C_p ln\frac{T_2}{T_1}\\
  S &= nC_V lnT + nRlnV +S_0 \Rightarrow \Delta S= C_V ln\frac{T_2}{T_1}\\
  p &= \frac{RT}{V_m-b}-\frac{a}{V_M^2}\\
  dW &= \sigma dA
\end{align}
\indent 声速
\begin{align}
  a &= \sqrt{\frac{dp}{d\rho}}\\
  a^2 &= \myPartial{p}{\rho}{S} = -v^2\myPartial{p}{v}{S}\\
  &= \gamma pv
\end{align}

\section{自由能}
\subsection{原型}
\begin{align}
  U &= TS - PV +\mu N\\
  H &= U + PV\\
  F &= U - TS\\
  G &= U - TS + PV\\
  \Omega &= U - TS - \mu N
\end{align}

\subsection{电}
\begin{align}
      \varepsilon (T) &= \frac{D}{E}\\
      dW &= VEdD\\
      p \rightarrow -E, V \rightarrow VD
\end{align}

\subsection{磁}
\begin{align}
  dW = \mu_0 \mathscr{H}d\mathscr{M}\\
  \dj\bard\dBar\dbar W = \mu_0 \mathscr{H}d\mathscr{M}\\
  p\rightarrow-\mu_0\mathscr{H}, V\rightarrow\mathscr{M}\\
  m = V\mathscr{M}\\
  \frac{C_{\mathscr{M}}{T}dT = \frac{C_V}{c}d\mathscr{M}
\end{align}



\subsection{微分形式}
\begin{align}
    dU &= TdS + \Sigma Y_i dy_i + \mu dN\\
    dH &= TdS+Vdp+\mu dN\\
    dF &= -SdT - PdV + \mu dN\\
    dG &= -SdT +Vdp+\mu dN\\
    d\Omega &= -SdT-PdV-Nd\mu\\
    Nd\mu &= -SdT +Vdp
\end{align}


\section{麦克斯韦关系}
\subsection{经典}
\begin{align}
  \bigg( \frac{\partial S}{\partial V}\bigg)_T = \bigg( \frac{\partial p}{\partial T}\bigg)_V,
  \bigg( \frac{\partial T}{\partial N}\bigg)_T = -\bigg( \frac{\partial \mu}{\partial S}\bigg)_N,
  \bigg( \frac{\partial p}{\partial N}\bigg)_{T,N} = -\bigg( \frac{\partial \mu}{\partial V}\bigg)_{T,V}
\end{align}
\subsection{扩展}
\begin{align}
    C_p - C_V &= T\bigg( \frac{\partial p}{\partial T}\bigg)\bigg( \frac{\partial V}{\partial T}\bigg)\\
              &= \frac{VT}{\kappa_T}\alpha^2\\
    C_V &= \myPartial{U}{T}{V}\\
    C_p &= \myPartial{H}{T}{p}\\
    \myPartial{U}{V}{T} &= T\myPartial{p}{T}{V}-p\\
    \myPartial{H}{p}{T} &= V-T\myPartial{V}{T}{p}\\
\end{align}
\indent吉布斯-亥姆霍兹方程
\begin{align}
  U =F - T\myPartial{F}{T}{} = G-T\myPartial{G}{T}{} -p\myPartial{G}{p}{}\\
  H = G-T\myPartial{G}{T}{}
\end{align}


\section{普朗克}
\begin{align}
  Q = \sigma T^4
\end{align}

\section{热力学平衡条件}
\begin{align}
    \delta^2 S = -\frac{C_V}{T^2}(\delta T)^2 + \frac{1}{T}\bigg( \frac{\partial p}{\partial V}\bigg)(\delta V)^2\\
    \Delta S<0,\Delta U>0,\Delta H>0,\Delta F>0,\Delta G>0
\end{align}

\section{相平衡}
\begin{align}
  Clapeyron \,\,equation:\frac{dp}{dT} = \frac{L}{T(V_m^{\beta}-V_m^{\alpha})}
\end{align}
\indent斯特林公式:
\begin{align}
  lnN! = NlnN - N
\end{align}
\indent相平衡共存
\begin{align}
  $热平衡条件:$ T^\alpha = T^\beta = T\\
  $力学平衡条件:$ p^\alpha = p^\beta = p\\
  $相变平衡条件:$ \mu^\alpha(T,p) = \mu^\beta(T,p)
\end{align}


\section{系综}
\subsection{欧拉定理}
\indent 欧拉定理:
\begin{align}
  if: f(\lambda x_1,\ldots,\lambda x_k) &= \lambda^mf(x^1,\ldots,x_k)\\
  then:\Sigma_i x_i\frac{\partial f}{\partial x_i} &= mf
\end{align}
\subsection{溶液蒸发点}
\begin{align}
  g' = g + RTln(1-x)
\end{align}
\subsection{熵}
\begin{align}
  S &= -k\Sigma_s \rho_s ln\rho_s,
  \rho_s =
    \begin{cases}
    \frac{1}{\Omega}& \text{微正则}\\
    \frac{1}{Z}e^{-\beta E_s}& \text{正则}\\
    \frac{1}{\Xi}e^{-\alpha N-\beta E_s}\\
    \end{cases}
\end{align}

\subsection{配分方程}
\subsubsection{正则(T,V,N)}
\begin{align}
  Z &= \frac{1}{N!h^{3N}}\int s^{-\beta E}dq_1\ldots dq_{3N}dp_1\ldots dp_{3N}\\
  \beta &= \frac{1}{kT}\\
  \rho_s &= \frac{1}{Z}e^{-\beta E_s}\\
  U &= -\frac{\partial}{\partial\beta}lnZ\\
  Y &= -\frac{1}{\beta}\frac{\partial}{\partial y}lnZ\\
  F &= -kTlnZ\\
  S &= k(lnZ - \beta\frac{\partial}{\partial\beta}lnZ)\\
  \mu &= -kT\frac{\partial}{\partial N}lnZ\\
  \overline{(E-\bar{E})^2} &= -\frac{\partial E}{\partial\beta} = kT^2C_V
\end{align}
\subsubsection{微正则(E,V,N)}
\begin{align}
  \rho_s &= \frac{1}{\Omega}\\
  \Omega &= \frac{1}{\prod_i N_i!h^{N_i r}}\int_{E\leq H(q,p)\leq E+\delta E} d\Omega\\
  \alpha &= \myPartial{ln\Omega_r}{N_r}{} = -\frac{\mu}{kT}\\
  \beta &= \myPartial{ln\Omega_r}{E_r}{}  = \frac{1}{kT}\\
  S(N,E,V) &= kln\Omega (N,E,V)\\
  dS&= \frac{1}{T}(pdV - \mu dN)\\
  T &= \bigg(\frac{\partial E}{\partial S}\bigg)_{V,N}\\
  p &= -\bigg(\frac{\partial E}{\partial V}\bigg)_{S,N}
\end{align}
\indent $p,\mu$都可以通过偏导求出。
\subsection{巨正则($\mu$,E,V)}
\begin{align}
  \Xi &= \Sigma_{N=0}^\infty \Sigma_s e^{-\alpha N -\beta E}\\
      &= \Sigma_{N}\frac{e^{-\alpha N}}{N!h^{Nr}}\int e^{-\beta E(q,p)}d\Omega\\
  \rho_{Ns} &= \frac{1}{\Xi}e^{-\alpha V-\beta E_s}\\
  \bar{N} &= -\frac{\partial}{\partial \alpha}ln\Xi\\
  U &= -\frac{\partial}{\partial \beta}ln\Xi\\
  p &= \frac{1}{\beta}\frac{\partial}{\partial V}ln\Xi\\
  Y &= -\frac{1}{\beta}\frac{\partial}{\partial y}ln\Xi\\
  S &= k(ln\Xi - \alpha \frac{\partial}{\partial \alpha}ln\Xi - \beta\frac{\partial}{\partial \beta}ln\Xi)\\
  \overline{(N-\bar{N})^2} &= kT\myPartial{\bar{N}}{\mu}{T,N}\\
  &= \frac{kT}{V}\kappa_T
\end{align}

\section{涨落}
\begin{align}
  W \propto e^{-\frac{\Delta S\Delta T - \Delta p\Delta V}{2kT}}\\
  \overline{\Delta T\cdot\Delta V} = \overline{\Delta T}\cdot\overline{\Delta V} = 0\\
  \overline{(\Delta T)^2} = \frac{kT^2}{C_V}\\
  \overline{(\Delta V)^2} = -kT\myPartial{V}{p}{T}
\end{align}
\indent 巨正则系综涨落
\begin{align}
    W \propto e^{-\frac{\Delta S\Delta T - \Delta p\Delta V +\Delta\mu\Delta N}{2kT}}
\end{align}
\indent 高斯分布
\begin{align}
  P(n) = \frac{1}{\sqrt{2\pi(\Delta n)^2}}e^{-\frac{(n-\bar{n})^2}{2\overline{(\Delta n)^2}}}
\end{align}


\section{蛋变鸡}
\indent鸡作为一个生物体并没有处在平衡态,因此不好用平常的方法衡量其熵的大小。然而我们考虑鸡蛋的混乱程度,其显然比鸡的混乱程度更大。



\end{spacing}
\end{CJK*}
\end{document}
